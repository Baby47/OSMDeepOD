%add new glossaryentries here...

\newglossaryentry{RQ}{
	name=RQ,
	description={RQ (Redis Queue) ist ist eine einfache Library für Python um Jobs in Redis einzureihen},
	first={RQ}
}

\newglossaryentry{MapQuest}{
	name=MapQuest,
	description={MapQuest ist eine API für OpenStreetMap Daten},
	first={MapQuest}
}

\newglossaryentry{Maptiler}{
	name=Maptiler,
	description={Maptiler bietet ein Python Skript zur Umrechnung von Koordinaten zu Tiles},
	first={Maptiler}
}

\newglossaryentry{Quadkeys}{
	name=Quadkeys,
	description={Quadkeys werden von Bing Maps für die Referenzierung zu ihren Tiles verwendet},
	first={Quadkeys}
}

\newglossaryentry{CircleCI}{
	name=CircleCI,
	description={CircleCI ist ein Tool für Continuous Integration und Deployment},
	first={CircleCI}
}

\newglossaryentry{Jira}{
	name=Jira,
	description={Jira ist eine webbasierte Anwendung für  Projektmanagement},
	first={Jira}
}

\newglossaryentry{PyCharm}{
	name=PyCharm,
	description={PyCharm ist eine IDE von JetBrains für Python},
	first={PyCharm}
}

\newglossaryentry{GitHub}{
	name=GitHub,
	description={GitHub ist ein webbasierter Software-Entwicklungsprojekte},
	first={GitHub}
}

\newglossaryentry{Git}{
	name=Git,
	description={Git ist ein Versionsverwaltungssystem},
	first={Git}
}

\newglossaryentry{OpenStreetMap}{
	name=OpenStreetMap,
	description={Online Karten Anwendung wie Google Maps},
	first={OpenStreetMap}
}

\newglossaryentry{Orthofotos}{
	name=Orthofotos,
	description={Luftbilder, umgangssprachlich Satellitenbilder},
	first={Orthofotos}
}

\newglossaryentry{Convolutional Neural Network}{
	name=Convolutional Neural Network,
	description={Neuronales Netz speziell für Bilderkennung},
	first={Convolutional Neural Network}
}

\newglossaryentry{Convnet}{
	name=Convnet,
	description={Siehe Convolutional Neural Network},
	first={Convnet}
}

\newglossaryentry{Deep Learning}{
	name=Deep Learning,
	description={Marketingbegriff für tiefe Neuronale Netze},
	first={Deep Learning}
}

\newglossaryentry{MapRoulette}{
	name=MapRoulette,
	description={Online Crowdsourcing-System zur spielerischen Überprüfung von Daten},
	first={MapRoulette}
}

\newglossaryentry{To-Fix}{
	name=ToFix,
	description={Online Crowdsourcing-System zur spielerischen Überprüfung von Daten},
	first={To-Fix}
}

\newglossaryentry{Docker}{
	name=Docker,
	description={Leichtgewichtige Visualisierung},
	first={Docker}
}

\newglossaryentry{Confusion Matrix}{
	name=Confusion Matrix,
	description={Wahrheitsmatrix, Methode zur Bewertung von Klassifizieren},
	first={Confusion Matrix}
}

\newglossaryentry{Haar Feature-based Cascade Classifier}{
	name=Haar Feature-based Cascade Classifier,
	description={Bilderklassifizierer aus OpenCV},
	first={Haar Feature-based Cascade Classifier}
}

\newglossaryentry{Scale-invariant Feature Transform}{
	name=Scale-invariant Feature Transform,
	description={Bildwiedererkenner aus OpenCV},
	first={Scale-invariant Feature Transform}
}

\newglossaryentry{Tile}{
	name=Tile,
	description={Quadratisches Bild mit Karteninhalt, Kachel},
	first={Tile}
}

\newglossaryentry{Bounding Box}{
	name=Bounding Box,
	description={Eine Fläche, welche durch 2 Längengrade und 2 Breitengrade definiert ist},
	first={Bounding Box}
}

\newglossaryentry{Keras}{
	name=Keras,
	description={Python Bibliothek zum Training und der Verwendung von Neuronalen Netzen},
	first={Keras}
}

\newglossaryentry{OpenCV}{
	name=OpenCV,
	description={C++ Bibliothek zur Verarbeitung von Bildern und Videos},
	first={OpenCV}
}

\newglossaryentry{Inputbild}{
	name=Inputbild,
	description={RGB Bild mit der Grösse 50 x 50 Pixel, welches als Input eines Convnet dient.},
	first={Inputbild}
}

\newglossaryentry{QuadTree}{
	name=QuadTree,
	description={Format zur Identifierung von Tiles},
	first={QuadTree}
}


