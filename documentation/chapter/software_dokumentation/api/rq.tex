
\subsection{Role Layer}

\Gls{RQ} \cite{RQ} (Redis Queue) ist eine einfache Library für Python um Jobs in Redis einzureihen. Im Anschluss sind Codebeispiele dazu angeführt.

\subsubsection{Beispiel Queue}
\begin{python}
from rq import Queue, use_connection
from redis import Redis

class RQueue(object):
   def __init__(self):
       redis = Redis(ip, port, password)
       q = Queue(connection=redis)
       job = q.enqueue(self.add, 2, 3)
       print job.result

   def add(self, x, y):
       return x + y
\end{python}

\subsubsection{Beispiel Worker}
\begin{python}
from rq import Queue, Connection, Worker
from redis import Redis

redis = Redis(ip, port, password)
with Connection(redis):
   qs = map(Queue, sys.argv[1:]) or [Queue()]
   w = Worker(qs)
   w.work()
\end{python}

\paragraph{Aufbau} Die Applikation beinhalten die Rollen:
\begin{itemize}
	\item Manger
	\begin{itemize}
		\item Teilt grosse Boundingbox in kleine auf.
		\item Stellt kleine Boundingboxes als Workerjobs in die Queue.
	\end{itemize}
		\item Jobworker
	\begin{itemize}
		\item Verarbeitet die Workerjobs.
		\item Führt den Erkennungsprozees durch.
		\item Stellt die Resultate als Resultjobs in die Queue.
	\end{itemize}
		\item Resultworker
	\begin{itemize}
		\item Verarbeite die Resultjobs.
		\item Speichert die Resultate in einem JSON File.
	\end{itemize}
\end{itemize}


\newpage
