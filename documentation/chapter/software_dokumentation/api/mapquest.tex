\subsection{MapQuest}
MapQuest\footnote{\url{http://www.mapquest.com/}} wird in diesem Projekt als Schnittstelle zu den OpenStreetMap Daten verwendet. Dazu bieten sich die Developer Accounts an, welche auf 15000 Abfragen pro Monat begrenzt sind, was unseren Abfrageumfang ausreichend abdeckt.

\subsubsection{Application Key}
In der Tabelle ist der Application Key aufgeführt, der für das Projekt Crosswalk Deteciton eingesetzt wurde.
\begin{table}[H]
    \begin{tabular}{ | p{6cm} | p{6cm}  | }
    \hline    
	Consumer Key &  YKqJ7JffQIBKyTgALLNXLVrDSaiQGtiI \\ \hline
	Consumer Secret & 3DO1eoLMxSqPH7Gk \\ \hline
	Key Issued & Fri, 09/25/2015 - 07:17 \\ \hline
	Key Expires & Never \\ \hline
    \end{tabular}
    \caption[MapQuest Application Key]{MapQuest Application Key}
\end{table}

\paragraph{Beispiel Abfragen}
Um den Entwicklern beim erstellen der Abfragen zu unterstützen wird folgende Webseite zur Verfügung gestellt:
\begin{itemize}
	\item \url{http://open.mapquestapi.com/xapi/}
\end{itemize}

\subparagraph{HTTP Request}
Bounding Box:  47.367,8.545,47.367,8.544 (Rapperswil)
\begin{itemize}
	\item \url{http://open.mapquestapi.com/xapi/api/0.6/node[highway=*][bbox=8.544,47.367,8.545,47.367]?key=YKqJ7JffQIBKyTgALLNXLVrDSaiQGtiI}
\end{itemize}


\subparagraph{XML File Response}
Als Antwort auf den HTTP Request gibt es ein XML File, welches die Strassen in Form von Nodes beinhaltet.\\

\begin{python}
<osm xmlns:xapi="http://jxapi.openstreetmap.org/">
	<node id="32860913" version="8"
		timestamp="2015-08-06T15:21:13Z" uid="6087"
		lat="47.2254172" lon="8.8175171">
		<tag k="highway" v="traffic_signals"/>
	</node>
</osm>
\end{python}

\newpage







