\chapter*{Abstract}

Zebrastreifen sind ein essentieller Bestandteil der Fussgängernavigation, diese sind jedoch nur spärlich erfasst, was zu nicht optimalen Routen führt. 
Um dem entgegen zu wirken,  befasst sich dieses Projekt mit der automatischen Erkennung von Zebrastreifen auf Orthofotos (Satellitenbildern). 
Dabei entstand eine Applikation, die auf den Orthofotos den Strassen folgt, diese in kleine Bilder unterteilt und mit Hilfe eines Deep learnig Ansatzes entscheidet, ob es sich um ein Zebrastreifen handelt oder nicht.
Das führte zu einer Erkennungsrate von über 85\% und könnte in Zukunft den Behöreden bei der Erfassung der Daten (derzeit noch händisch) unterstützen.
Weiter ist es möglich diese Lösung auszubauen und auf andere Objekte anzuwenden.
