\section{Abstract}

Zebrastreifen sind ein essentieller Bestandteil der Fussgängernavigation. Sie sind jedoch  nur spärlich erfasst, was somit zu nicht optimalen Routen führt. Ohne Fussgängerstreifen ist keine Strassenüberquerung möglich.

Um dem entgegen zu wirken,  befasst sich dieses Projekt mit der automatischen Erkennung von Zebrastreifen auf Orthofotos (Satellitenbildern). 
Dabei entstand eine hoch skalierbare Applikation, die gelbe Fussgängerstreifen entlang von Strassen findet und ihre Koordinaten extrahiert. Die Erkennung erfolgt durch ein Convolutional Neural Network, welches im Bereich Deep Learning anzusiedeln ist.
Das Projekt erreichte eine Erkennungsrate von über 85\% der visuell sichtbaren Fussgängerstreifen mit einer Fehlerrate von weniger als 10\%. Der Erkennungsprozess konnte mithilfe von Message Queuing auf eine grosse Anzahl von Computern verteilt werden. Die Wartezeit für die Erkennung konnte so auf mehrere Tage beschränkt werden. Die Koordinaten der Fussgängerstreifen wurden an MapRoulette übergeben.

Es ist möglich diese Lösung so auszubauen, dass es auf weitere Objekte angewendet werden kann.
