\section{Abstract}

Fussgängerstreifen sind ein essentieller Bestandteil der Fussgängernavigation. Sie sind jedoch nur spärlich in OpenStreetMap erfasst, was somit zu nicht optimalen Routen führt.

Um dem entgegen zu wirken,  befasst sich dieses Projekt mit der automatischen Erkennung von Fussgängerstreifen auf Orthofotos (Satellitenbildern). 
Dabei entstand eine hoch skalierbare Applikation, die gelbe Fussgängerstreifen entlang von Strassen findet und ihre Koordinaten extrahiert. Die Erkennung erfolgt durch ein Convolutional Neural Network, welches im Bereich des Deep Learnings angesiedelt ist.
Das Projekt erreichte eine Erkennungsrate von über 80\% der visuell sichtbaren Fussgängerstreifen mit einer Fehlerrate von weniger als 10\%. Der Erkennungsprozess konnte mit Hilfe eines Queueing Systems auf eine grosse Anzahl von Computern verteilt werden, was die Verarbeitung so vieler Daten in angemessener Zeit überhaupt erst ermöglichte. Die Koordinaten der Fussgängerstreifen wurden im Anschluss an das Crowdsourcing-System MapRoulette übergeben. Auf diesem Weg konnten die Daten in OpenStreetMap integriert werden und zur Verbesserung der Fussgängernavigation ihren Anteil beitragen.

Es ist möglich diese Lösung so auszubauen, dass es auf weitere Objekte angewendet werden kann.
