\section{Planung}
Am Anfang es Projektes haben wir eine grobe Planung zusammengestellt. Dabei haben wir die Phasen und Meilensteine definiert. Während dem Projekt stellten wir immer wieder grössere oder kleinere Abweichungen an der zu Beginn definierten Planung fest. Dieses ist jedoch nicht erstaunlich, da nie absolut korrekt geplant werden kann. Um solche Schwierigkeiten zu handhaben, erstellten wir ein Risikomanagementdokument.

\subsection{Phasen}
\begin{enumerate}
  \item Inception
  \begin{enumerate}
    \item Aufgabenstellung ausarbeiten
  \end{enumerate}
  \item Elaboration1
  \begin{enumerate}
    \item Evaluation der Algorithmen (Bilderkennung)
  \end{enumerate}
  \item Elaboration2
  \begin{enumerate}
    \item Prototyp 1 (in Orthofotos, out Koordinaten)
    \item Prototyp 2 MapRoulett
  \end{enumerate}
  \item Construction1
  \begin{enumerate}
    \item Schnittstelle Satellitenbilder
    \item Optimierungen durch Strassenverlauf und ähnliches
  \end{enumerate}
  \item Construction2
  \begin{enumerate}
    \item Maproulette (Tags und Quiz)
    \item Koordinaten erfassen	
  \end{enumerate}
  \item Transition
  \begin{enumerate}
    \item Dokumentation abschliessen
    \item Challenge auf Maproulette
  \end{enumerate}
\end{enumerate}
\newpage

\subsection{Meilensteine}
\begin{enumerate}
	\item MS1 Algorithmus für Bilderkennung evaluiert
	\item MS2 Prototyp erstellt
	\item MS3 Automatisierte Datenverarbeitung
	\item MS4 Applikation fertiggestellt
	\item MS5 Challenge auf Maproulette
\end{enumerate}

\subsection{Zeitplanung}
Aufwand: 14 Wochen zu 2 * 16 Stunden = \textbf{448 Stunden}
\begin{tabbing}[H]
    \hspace*{6cm}\=\hspace*{6cm}\= \kill
    Inception \>  1 Woche \\
	Elaboration1 \>	3 Wochen \\
	Elaboration2 \>	4 Wochen \\
	Construction1 \> 3 Wochen \\
	Construction2 \> 1 Wochen \\
	Transition \> 2 Wochen \\
\end{tabbing}

