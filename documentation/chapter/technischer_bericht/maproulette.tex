\subsection{MapRoulette}
Maproulette ist ein Crowdsourcing-System, welches genutzt wird um mit Hilfe von Challenges Daten in OpenStreetMap einfliessen zu lassen, diese kontrolliert oder auch erweitert. Den Entscheid für diesen Anbieter haben wir unter dem Abschnitt~\ref{subsec:MapRoulette} auf der Seite~\pageref{subsec:MapRoulette} getroffen.

\subsubsection{Daten in Tasks umwandeln}
Am Ende des Suchprozesses werden alle gefundenen Fussgängerstreifen in einer JSON Datei abgespeichert. MapRoulette benötigt für die Tasks jedoch ein GeoJSON Format, somit mussten wir diese Daten noch passend umwandeln. 

\paragraph{Challenge} Der Text für die Challenge ist in englisch zu verfassen und sollte folgende Punkte beinhalten:
\begin{itemize}
	\item Title
	\item Blurb (Einzeiler, eine Art Untertitel)
	\item Description
	\item Help
	\item Instruction
\end{itemize}

\paragraph{Task} Unsere Task beinhalte hauptsächlich nur die Position des zu verifizierenden Fussgängerstreifens, sowie eine kurze Beschreibung. Der Aufbau gliedert sich folgendermassen:
\begin{itemize}
	\item Identifier (Eindeutiger 72 Character String)
	\item Geometries (Position des Fussgängerstreifens)
	\item Instruction (Beschriebt den Task)
\end{itemize}