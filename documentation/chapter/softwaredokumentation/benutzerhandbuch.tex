\section{Benutzerhandbuch}
\subsection{Suche der Fussgängerstreifen}
Um die Suche der Fussgängerstreifen durchzuführen muss eine Redis Datenbank zur Verfügung stehen , weiter muss auf den Rechnern die als Jobworker tätig sind Docker installiert sein. \\
Die Installationen sind in folgendne Abschnitten aufgeführt:
\begin{itemize}
	\item Redis:  Abschnitt~\ref{subsec:redis} auf der Seite~\pageref{subsec:redis}
	\item Docker: Abschnitt~\ref{subsec:docker} auf der Seite~\pageref{subsec:docker}
\end{itemize}

\subsubsection{Einführung}
Wir haben unsere Applikation in drei Rollen aufgeteilt:
\begin{itemize}
	\item Manager
	\begin{itemize}
		\item Unterteilt eine grosse Boundingbox in Kleinere mit einer Höhe und Breite von jeweils 2 Kilometern und stellt dies als Jobs in die Queue.
	\end{itemize}
	\item Jobworker
	\begin{itemize}
		\item Arbeite die Jobs der Queue ab
		\item Gefundene Fussgängerstreifen , welche noch nicht in OpenSteetMap erfasst sind, werden als "Resultat Job" in die Queue gestellt
	\end{itemize}
	\item Resultworker
	\begin{itemize}
		\item Schlussendlich werden die Resultate zusammen getragen und in ein JSON File geschrieben
	\end{itemize}
\end{itemize}

Dieser Ablauf ist genauer beschrieben unter dem Abschnitt~\ref{subsec:ablauf} auf   auf der Seite~\pageref{subsec:ablauf}
\newpage
\subsubsection{Anwendung}
Dank Docker kann unsere Applikation innert Minuten gestartet werden.

\paragraph{Docker Pull}
\begin{lstlisting}[style=BashInputStyle]
	# docker pull murthy10/osm-crosswalk-detection
\end{lstlisting}

\paragraph{Manager}
\begin{lstlisting}[style=BashInputStyle]
 # docker run murthy10/osm-crosswalk-detection REDIS_IP_ADDR 
 --role manager left bottom right top
\end{lstlisting}
Left, Bottom, Right, Top entsprechen den Koordinaten im WGS84 Format. \\
Ostschweiz: left=8.361002, bottom=47.166994, right=8.977610, top=47.706676) \\

\paragraph{Jobworker}
\begin{lstlisting}[style=BashInputStyle]
 # docker run murthy10/osm-crosswalk-detection REDIS_IP_ADDR 
 --role jobworker
\end{lstlisting}
Jobworker können auf bliebig vielen Rechnern gestartet werden.\\

\paragraph{Resultworker}
\begin{lstlisting}[style=BashInputStyle]
 # docker run murthy10/osm-crosswalk-detection REDIS_IP_ADDR 
 --role resultworker
\end{lstlisting}
Die Resultate werden im File crosswalks.json gespeichert, dieses findet man im Verzeichnis in dem der Resultworker gestartet wurde.

